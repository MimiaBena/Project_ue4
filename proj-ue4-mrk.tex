\documentclass[]{article}
\usepackage{lmodern}
\usepackage{amssymb,amsmath}
\usepackage{ifxetex,ifluatex}
\usepackage{fixltx2e} % provides \textsubscript
\ifnum 0\ifxetex 1\fi\ifluatex 1\fi=0 % if pdftex
  \usepackage[T1]{fontenc}
  \usepackage[utf8]{inputenc}
\else % if luatex or xelatex
  \ifxetex
    \usepackage{mathspec}
  \else
    \usepackage{fontspec}
  \fi
  \defaultfontfeatures{Ligatures=TeX,Scale=MatchLowercase}
\fi
% use upquote if available, for straight quotes in verbatim environments
\IfFileExists{upquote.sty}{\usepackage{upquote}}{}
% use microtype if available
\IfFileExists{microtype.sty}{%
\usepackage{microtype}
\UseMicrotypeSet[protrusion]{basicmath} % disable protrusion for tt fonts
}{}
\usepackage[margin=1in]{geometry}
\usepackage{hyperref}
\hypersetup{unicode=true,
            pdftitle={proj-ue4},
            pdfborder={0 0 0},
            breaklinks=true}
\urlstyle{same}  % don't use monospace font for urls
\usepackage{color}
\usepackage{fancyvrb}
\newcommand{\VerbBar}{|}
\newcommand{\VERB}{\Verb[commandchars=\\\{\}]}
\DefineVerbatimEnvironment{Highlighting}{Verbatim}{commandchars=\\\{\}}
% Add ',fontsize=\small' for more characters per line
\usepackage{framed}
\definecolor{shadecolor}{RGB}{248,248,248}
\newenvironment{Shaded}{\begin{snugshade}}{\end{snugshade}}
\newcommand{\KeywordTok}[1]{\textcolor[rgb]{0.13,0.29,0.53}{\textbf{#1}}}
\newcommand{\DataTypeTok}[1]{\textcolor[rgb]{0.13,0.29,0.53}{#1}}
\newcommand{\DecValTok}[1]{\textcolor[rgb]{0.00,0.00,0.81}{#1}}
\newcommand{\BaseNTok}[1]{\textcolor[rgb]{0.00,0.00,0.81}{#1}}
\newcommand{\FloatTok}[1]{\textcolor[rgb]{0.00,0.00,0.81}{#1}}
\newcommand{\ConstantTok}[1]{\textcolor[rgb]{0.00,0.00,0.00}{#1}}
\newcommand{\CharTok}[1]{\textcolor[rgb]{0.31,0.60,0.02}{#1}}
\newcommand{\SpecialCharTok}[1]{\textcolor[rgb]{0.00,0.00,0.00}{#1}}
\newcommand{\StringTok}[1]{\textcolor[rgb]{0.31,0.60,0.02}{#1}}
\newcommand{\VerbatimStringTok}[1]{\textcolor[rgb]{0.31,0.60,0.02}{#1}}
\newcommand{\SpecialStringTok}[1]{\textcolor[rgb]{0.31,0.60,0.02}{#1}}
\newcommand{\ImportTok}[1]{#1}
\newcommand{\CommentTok}[1]{\textcolor[rgb]{0.56,0.35,0.01}{\textit{#1}}}
\newcommand{\DocumentationTok}[1]{\textcolor[rgb]{0.56,0.35,0.01}{\textbf{\textit{#1}}}}
\newcommand{\AnnotationTok}[1]{\textcolor[rgb]{0.56,0.35,0.01}{\textbf{\textit{#1}}}}
\newcommand{\CommentVarTok}[1]{\textcolor[rgb]{0.56,0.35,0.01}{\textbf{\textit{#1}}}}
\newcommand{\OtherTok}[1]{\textcolor[rgb]{0.56,0.35,0.01}{#1}}
\newcommand{\FunctionTok}[1]{\textcolor[rgb]{0.00,0.00,0.00}{#1}}
\newcommand{\VariableTok}[1]{\textcolor[rgb]{0.00,0.00,0.00}{#1}}
\newcommand{\ControlFlowTok}[1]{\textcolor[rgb]{0.13,0.29,0.53}{\textbf{#1}}}
\newcommand{\OperatorTok}[1]{\textcolor[rgb]{0.81,0.36,0.00}{\textbf{#1}}}
\newcommand{\BuiltInTok}[1]{#1}
\newcommand{\ExtensionTok}[1]{#1}
\newcommand{\PreprocessorTok}[1]{\textcolor[rgb]{0.56,0.35,0.01}{\textit{#1}}}
\newcommand{\AttributeTok}[1]{\textcolor[rgb]{0.77,0.63,0.00}{#1}}
\newcommand{\RegionMarkerTok}[1]{#1}
\newcommand{\InformationTok}[1]{\textcolor[rgb]{0.56,0.35,0.01}{\textbf{\textit{#1}}}}
\newcommand{\WarningTok}[1]{\textcolor[rgb]{0.56,0.35,0.01}{\textbf{\textit{#1}}}}
\newcommand{\AlertTok}[1]{\textcolor[rgb]{0.94,0.16,0.16}{#1}}
\newcommand{\ErrorTok}[1]{\textcolor[rgb]{0.64,0.00,0.00}{\textbf{#1}}}
\newcommand{\NormalTok}[1]{#1}
\usepackage{graphicx,grffile}
\makeatletter
\def\maxwidth{\ifdim\Gin@nat@width>\linewidth\linewidth\else\Gin@nat@width\fi}
\def\maxheight{\ifdim\Gin@nat@height>\textheight\textheight\else\Gin@nat@height\fi}
\makeatother
% Scale images if necessary, so that they will not overflow the page
% margins by default, and it is still possible to overwrite the defaults
% using explicit options in \includegraphics[width, height, ...]{}
\setkeys{Gin}{width=\maxwidth,height=\maxheight,keepaspectratio}
\IfFileExists{parskip.sty}{%
\usepackage{parskip}
}{% else
\setlength{\parindent}{0pt}
\setlength{\parskip}{6pt plus 2pt minus 1pt}
}
\setlength{\emergencystretch}{3em}  % prevent overfull lines
\providecommand{\tightlist}{%
  \setlength{\itemsep}{0pt}\setlength{\parskip}{0pt}}
\setcounter{secnumdepth}{0}
% Redefines (sub)paragraphs to behave more like sections
\ifx\paragraph\undefined\else
\let\oldparagraph\paragraph
\renewcommand{\paragraph}[1]{\oldparagraph{#1}\mbox{}}
\fi
\ifx\subparagraph\undefined\else
\let\oldsubparagraph\subparagraph
\renewcommand{\subparagraph}[1]{\oldsubparagraph{#1}\mbox{}}
\fi

%%% Use protect on footnotes to avoid problems with footnotes in titles
\let\rmarkdownfootnote\footnote%
\def\footnote{\protect\rmarkdownfootnote}

%%% Change title format to be more compact
\usepackage{titling}

% Create subtitle command for use in maketitle
\newcommand{\subtitle}[1]{
  \posttitle{
    \begin{center}\large#1\end{center}
    }
}

\setlength{\droptitle}{-2em}

  \title{proj-ue4}
    \pretitle{\vspace{\droptitle}\centering\huge}
  \posttitle{\par}
    \author{}
    \preauthor{}\postauthor{}
    \date{}
    \predate{}\postdate{}
  

\begin{document}
\maketitle

\subsection{1-Introduction :}\label{introduction}

La fouille de données (data mining) désigne l'analyse de données depuis
différentes perspectives et le fait de transformer ces données en
informations utiles, alors c'est une discipline qui permet de faire un
lien entre les statistiques et les technologies de l'information (base
de données, intelligence artificielle, apprentissage automatique
(machine learning). Au début, la fouille de données était utilisée dans
la gestion de la relation client, pour mieux les fidéliser et leur
proposer des produits qui leur sont adaptés. Aujourd'hui, la recherche
d'informations dans les grandes bases de données médicales ou de santé
se développe de plus en plus . Les outils de collecte automatique des
données et bases de données permettent de stocker dans des entrepôts
d'énormes masses de données. La fouille de données et les entrepôts
permettent l'extraction de connaissances. \#\# 2- Matériels et méthodes
\#\#\# Outils utilisées 1.1.1- R et Rstudio R est un logiciel permettant
de faire des analyses statistiques et permet la production de graphiques
évolués. C'est aussi un langage de programmation,. Fonctionnant en
console, son utilisation n'est pas aisée. L'interface Rstudio permet une
utilisation de R les fonctionnalités de R plus facilement. 1.1.2-
Resources cartographiques Pour pouvoir créer une cartographie sur R, il
nous a fallu importer des fonds de cartes.

\begin{Shaded}
\begin{Highlighting}[]
\KeywordTok{library}\NormalTok{(markdown)}
\KeywordTok{library}\NormalTok{(shiny)}
\KeywordTok{options}\NormalTok{(}\DataTypeTok{repos =} \KeywordTok{c}\NormalTok{(}\DataTypeTok{CRAN =} \StringTok{"http://cran.rstudio.com"}\NormalTok{))}
\KeywordTok{library}\NormalTok{(dplyr)}
\end{Highlighting}
\end{Shaded}

\begin{verbatim}
## 
## Attaching package: 'dplyr'
\end{verbatim}

\begin{verbatim}
## The following objects are masked from 'package:stats':
## 
##     filter, lag
\end{verbatim}

\begin{verbatim}
## The following objects are masked from 'package:base':
## 
##     intersect, setdiff, setequal, union
\end{verbatim}

\begin{Shaded}
\begin{Highlighting}[]
\KeywordTok{library}\NormalTok{(sp)}
\KeywordTok{library}\NormalTok{(rgdal)}
\end{Highlighting}
\end{Shaded}

\begin{verbatim}
## rgdal: version: 1.3-6, (SVN revision 773)
##  Geospatial Data Abstraction Library extensions to R successfully loaded
##  Loaded GDAL runtime: GDAL 2.2.3, released 2017/11/20
##  Path to GDAL shared files: C:/Program Files/R/R-3.5.2/library/rgdal/gdal
##  GDAL binary built with GEOS: TRUE 
##  Loaded PROJ.4 runtime: Rel. 4.9.3, 15 August 2016, [PJ_VERSION: 493]
##  Path to PROJ.4 shared files: C:/Program Files/R/R-3.5.2/library/rgdal/proj
##  Linking to sp version: 1.3-1
\end{verbatim}

\begin{Shaded}
\begin{Highlighting}[]
\KeywordTok{library}\NormalTok{(tidyr)}
\KeywordTok{library}\NormalTok{(mapproj)}
\end{Highlighting}
\end{Shaded}

\begin{verbatim}
## Loading required package: maps
\end{verbatim}

\begin{Shaded}
\begin{Highlighting}[]
\KeywordTok{library}\NormalTok{(maps)}
\end{Highlighting}
\end{Shaded}

\begin{Shaded}
\begin{Highlighting}[]
\CommentTok{#Téléchargements des données}
\NormalTok{radon <-}\StringTok{ 'C:/Users/Mimia/Pictures/Documents/projet-ue4/radon.csv'}
\NormalTok{code_postal <-}\StringTok{ 'C:/Users/Mimia/Pictures/Documents/projet-ue4/code-postal-code-insee-2015.csv'}

\NormalTok{catalogue_radon <-}\StringTok{ }\KeywordTok{read.csv}\NormalTok{(radon, }\DataTypeTok{sep=}\StringTok{";"}\NormalTok{,}\DataTypeTok{encoding=}\StringTok{'UTF-8'}\NormalTok{)}
\NormalTok{catalogue_code <-}\StringTok{ }\KeywordTok{read.csv}\NormalTok{(code_postal, }\DataTypeTok{sep=}\StringTok{";"}\NormalTok{,}\DataTypeTok{encoding=}\StringTok{'UTF-8'}\NormalTok{)}
\end{Highlighting}
\end{Shaded}

\begin{Shaded}
\begin{Highlighting}[]
\NormalTok{departements<-}\StringTok{ }\KeywordTok{readOGR}\NormalTok{(}\StringTok{"C:/Users/Mimia/Pictures/Documents/projet-ue4/DEPARTEMENT.shp"}\NormalTok{)}
\end{Highlighting}
\end{Shaded}

\begin{verbatim}
## OGR data source with driver: ESRI Shapefile 
## Source: "C:\Users\Mimia\Pictures\Documents\projet-ue4\DEPARTEMENT.shp", layer: "DEPARTEMENT"
## with 96 features
## It has 11 fields
\end{verbatim}

\begin{Shaded}
\begin{Highlighting}[]
 \KeywordTok{summary}\NormalTok{(departements)}
\end{Highlighting}
\end{Shaded}

\begin{verbatim}
## Object of class SpatialPolygonsDataFrame
## Coordinates:
##         min     max
## x   99217.1 1242417
## y 6049646.3 7110480
## Is projected: TRUE 
## proj4string :
## [+proj=lcc +lat_1=44 +lat_2=49 +lat_0=46.5 +lon_0=3 +x_0=700000
## +y_0=6600000 +ellps=GRS80 +units=m +no_defs]
## Data attributes:
##                     ID_GEOFLA    CODE_DEPT                     NOM_DEPT 
##  DEPARTEM0000000000000001: 1   01     : 1   AIN                    : 1  
##  DEPARTEM0000000000000002: 1   02     : 1   AISNE                  : 1  
##  DEPARTEM0000000000000003: 1   03     : 1   ALLIER                 : 1  
##  DEPARTEM0000000000000004: 1   04     : 1   ALPES-DE-HAUTE-PROVENCE: 1  
##  DEPARTEM0000000000000005: 1   05     : 1   ALPES-MARITIMES        : 1  
##  DEPARTEM0000000000000006: 1   06     : 1   ARDECHE                : 1  
##  (Other)                 :90   (Other):90   (Other)                :90  
##     CODE_CHF     NOM_CHF     X_CHF_LIEU        Y_CHF_LIEU     
##  001    : 2   AGEN   : 1   Min.   : 171326   Min.   :6108968  
##  004    : 2   AJACCIO: 1   1st Qu.: 553177   1st Qu.:6401238  
##  007    : 2   ALBI   : 1   Median : 658409   Median :6631771  
##  010    : 2   ALENCON: 1   Mean   : 684049   Mean   :6609885  
##  033    : 2   AMIENS : 1   3rd Qu.: 846156   3rd Qu.:6820420  
##  085    : 2   ANGERS : 1   Max.   :1228512   Max.   :7059443  
##  (Other):84   (Other):90                                      
##    X_CENTROID        Y_CENTROID         CODE_REG 
##  Min.   : 115669   Min.   :6050394   76     :13  
##  1st Qu.: 544294   1st Qu.:6398674   75     :12  
##  Median : 659892   Median :6639324   84     :12  
##  Mean   : 681127   Mean   :6611276   44     :10  
##  3rd Qu.: 838334   3rd Qu.:6834668   11     : 8  
##  Max.   :1225168   Max.   :7046488   27     : 8  
##                                      (Other):33  
##                                 NOM_REG  
##  LANGUEDOC-ROUSSILLON-MIDI-PYRENEES :13  
##  AQUITAINE-LIMOUSIN-POITOU-CHARENTES:12  
##  AUVERGNE-RHONE-ALPES               :12  
##  ALSACE-CHAMPAGNE-ARDENNE-LORRAINE  :10  
##  BOURGOGNE-FRANCHE-COMTE            : 8  
##  ILE-DE-FRANCE                      : 8  
##  (Other)                            :33
\end{verbatim}

\begin{Shaded}
\begin{Highlighting}[]
\CommentTok{# fusionne les data frames x et y par la colonne name.}
\NormalTok{radon_code <-}\KeywordTok{merge}\NormalTok{(catalogue_radon, catalogue_code, }\DataTypeTok{by =}\StringTok{"insee_com"}\NormalTok{)}
\end{Highlighting}
\end{Shaded}

\begin{Shaded}
\begin{Highlighting}[]
\CommentTok{#créer les 2coordonnées}
\NormalTok{radon_code<-}\StringTok{  }\NormalTok{radon_code  }\OperatorTok
\StringTok{  }\KeywordTok{separate}\NormalTok{(}\DataTypeTok{col =} \StringTok{"coordonnees_gps"}\NormalTok{,}
           \DataTypeTok{into =} \KeywordTok{paste0}\NormalTok{(}\StringTok{"coordonnees_gps"}\NormalTok{,}\DecValTok{1}\OperatorTok{:}\DecValTok{2}\NormalTok{), }\DataTypeTok{sep =} \StringTok{","}\NormalTok{,}
           \DataTypeTok{extra =} \StringTok{"merge"}\NormalTok{)}
\end{Highlighting}
\end{Shaded}

\begin{Shaded}
\begin{Highlighting}[]
\NormalTok{## Sélection de certaines colonnes}
\NormalTok{radon_code <-}\StringTok{ }\NormalTok{radon_code }\OperatorTok\StringTok{ }
\StringTok{  }\KeywordTok{select}\NormalTok{(insee_com,nom_comm, nom_dept, }
\NormalTok{                            classe_potentiel,coordonnees_gps1,coordonnees_gps2) }\OperatorTok
\StringTok{  }\KeywordTok{unique}\NormalTok{()}
\end{Highlighting}
\end{Shaded}

\begin{Shaded}
\begin{Highlighting}[]
\NormalTok{## Jointure avec le fond de carte pour obtenir les nom de pays}
\end{Highlighting}
\end{Shaded}

\begin{Shaded}
\begin{Highlighting}[]
\NormalTok{      test<-}\KeywordTok{match}\NormalTok{(departements}\OperatorTok{$}\NormalTok{NOM_COM,radon_code}\OperatorTok{$}\NormalTok{nom_comm)}
\NormalTok{      couleurs<-radon_code}\OperatorTok{$}\NormalTok{classe_potentiel[test]}
      
\NormalTok{      getColor <-}\StringTok{ }\ControlFlowTok{function}\NormalTok{(test) \{}
        \KeywordTok{sapply}\NormalTok{(test}\OperatorTok{$}\NormalTok{classe_potentiel, }\ControlFlowTok{function}\NormalTok{(classe_potentiel) \{}
          \ControlFlowTok{if}\NormalTok{(classe_potentiel}\OperatorTok{==}\StringTok{ }\DecValTok{1}\NormalTok{) \{}
            \StringTok{"green"}
\NormalTok{          \} }\ControlFlowTok{else} \ControlFlowTok{if}\NormalTok{(classe_potentiel }\OperatorTok{==}\DecValTok{2}\NormalTok{) \{}
            \StringTok{"orange"}
\NormalTok{          \} }\ControlFlowTok{else}\NormalTok{ \{}
            \StringTok{"red"}
\NormalTok{          \} \})}
\NormalTok{      \}}
\end{Highlighting}
\end{Shaded}

\begin{Shaded}
\begin{Highlighting}[]
\NormalTok{ui <-}\StringTok{ }\KeywordTok{fluidPage}\NormalTok{(}
        \KeywordTok{titlePanel}\NormalTok{(}\StringTok{"app"}\NormalTok{),}
        \KeywordTok{sidebarLayout}\NormalTok{(}
          \KeywordTok{sidebarPanel}\NormalTok{(}
            \KeywordTok{fluidRow}\NormalTok{(}\KeywordTok{selectInput}\NormalTok{(}\StringTok{"select"}\NormalTok{,}\StringTok{"select a snp"}\NormalTok{,}\DataTypeTok{choices =} \DecValTok{1}\NormalTok{)}
\NormalTok{            )}
\NormalTok{          ),}
          \KeywordTok{mainPanel}\NormalTok{( }\KeywordTok{plotOutput}\NormalTok{(}\StringTok{"map"}\NormalTok{))}
\NormalTok{        )}
\NormalTok{      )}
\end{Highlighting}
\end{Shaded}

\begin{Shaded}
\begin{Highlighting}[]
\CommentTok{# Server logic ----}
\NormalTok{      server <-}\StringTok{ }\ControlFlowTok{function}\NormalTok{(input, output) \{}
\NormalTok{        output}\OperatorTok{$}\NormalTok{map <-}\StringTok{ }\KeywordTok{renderPlot}\NormalTok{(\{}
          \KeywordTok{plot}\NormalTok{(departements, }\DataTypeTok{fill=}\OtherTok{TRUE}\NormalTok{,}\DataTypeTok{col=}\KeywordTok{getColor}\NormalTok{(radon_code))}
\NormalTok{        \})}
\NormalTok{      \}}
\end{Highlighting}
\end{Shaded}

\begin{Shaded}
\begin{Highlighting}[]
 \CommentTok{# Run app ----}
      \KeywordTok{shinyApp}\NormalTok{(ui, server)}
\end{Highlighting}
\end{Shaded}

\begin{verbatim}
## PhantomJS not found. You can install it with webshot::install_phantomjs(). If it is installed, please make sure the phantomjs executable can be found via the PATH variable.
\end{verbatim}

Shiny applications not supported in static R Markdown documents


\end{document}
